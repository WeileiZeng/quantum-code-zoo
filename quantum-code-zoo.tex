\def\rank{\mathop{\rm rank}}
\def\wgt{\mathop{\rm wgt}}
\def\lc{\mathop{\rm lc}}
\def\ker{\mathop{\rm Ker}}
\def\im{\mathop{\rm Im}}
\documentclass[aps,%prb,12pt,tightenlines,%
pra, twocolumn,%superscriptaddress,%
%galley,
notitlepage,longbibliography]{revtex4-2}
%\advance\textheight by -5.2in
%\advance\textheight by -2.25in
\usepackage{hyperref}
\usepackage{amsmath}
%\usepackage{graphicx}
\usepackage{xcolor}
\usepackage{amsthm}
\usepackage{amsfonts}
\usepackage{amssymb}
%\usepackage{array}
%\usepackage{enumitem}
%\usepackage{datetime}
\usepackage[ampersand]{easylist}
\ListProperties(Hide=100, Hang=true, Progressive=3ex, Style*=-- ,
Style2*=$\bullet$ ,Style3*=$\circ$ ,Style4*=\tiny$\blacksquare$ )
% ...

\newtheorem{theorem}{Theorem}
\newtheorem{example}[theorem]{Example}
\newtheorem{note}[theorem]{Note}
\newtheorem{corollary}[theorem]{Corollary}
\newtheorem{lemma}[theorem]{Lemma}
\newtheorem{definition}[theorem]{Definition}
\begin{document}
\title{Quantum Code Zoo}
\date\today
\date{\today\ \bf \jobname} 
%\date{\today\ \currenttime\ \bf \jobname} 
\author{ZWL}
%\affiliation{Xanadu, Toronto, ON, M5G 2C8, Canada}
\email{weilei.zeng@foxmail.com}

\begin{abstract}
  A zoo of quantum codes, inspired by the quantum algorithm zoo. This
  zoo includes a classification of all known codes and an one-liner
  description with construction reference. This is an ongoing project,
  contributions, pull requests and comments are welcome.
\end{abstract}
\maketitle

\tableofcontents


\section{Introduction}
Since the discovery of Shor's codes in 1995 [cite], quantum error
correction has experienced fast development in the past two
decades. Still, new codes comes out week by week, with starling
progress. It is a non-trivial task to give a proper name for a new
class of code. A good name gives pictural description of the code,
providing no confusion and receiving no complain.

One can find a discussion of names used for physics quantities and theorems
here [cite]





\section{Quantum code zoo}


Each code is classified into one or more categories in this list.



Whatever code you find, let's find a place for it in this list
\begin{easylist}
  & fermion codes
  
  & bosonic codes
  && GKP codes
  && cat codes
  
  & CWS
  && Stabilizer codes
  
  &&& CSS codes
  &&&& QHP codes
  &&&&& toric codes
  &&&&& HQHP codes
  &&&&&& toric codes in higher-Dimension

  &&&& Quantum bicycle codes
  &&&& Homological product codes
  &&&& Lifted product codes
  &&&& Fiber bundle codes
  &&&& Quantum pin codes
  &&&& 
  
  &&& non-CSS codes
  &&&& rotated surface codes
  &&&& Quantum XYZ product codes


  && Subsystem codes
  &&& Subsystem product codes
  &&&& Subsystem hypergraph product codes
  &&&&& Bacon Shor codes



  && Concatenated codes
  &&& Shor's codes
  
\end{easylist}

unclassified
\begin{easylist}
  & quantum hyperbicycle codes\cite{kovalev1212quantum}
  
\end{easylist}
  
\section{Glossary and references}
This section has a one-liner explanation for each codes, plus
necessary notes, in alphabet order

Format

code name [cite]: description + necessary notes

Bacon Shor codes

BCH codes

Binomial codes

Bosonic codes

Cat codes

CSS codes: Two classical codes that one contains the other; A
stabilizer code with X-type and Z-type check operators.

Concatenated cat codes: Concatenated codes with cat code as the inner
code and another qubit code as the outer code

Concatenated codes: A multi-layer structure where the logical qubits
of one code are used as the physical qubits of another code.


Color codes:

Cubic codes: toric codes in 3D

Data syndrome codes: when measurement error are considered, it adds
extra bits to the code, hence called data syndrome codes. It is
similar to space-time codes.

Fiber bundle codes

Gottesman-Kitaev-Preskill (GKP) codes \cite{gottesman2001encoding}: encode a qubit into an
oscillator, that is, continuous variable. It could be a qudit as well.


Higher-dimensional quantum hypergraph product codes:

Homological product codes

Hypercubic codes: toric codes in 4D


Lifted product codes

Stabilizer codes: A subspace stabilized by an abelian subgroup of the
Pauli group.


Quantum bicycle codes:

Quantum convolutional codes:

Quantum Hamming codes:
A $[[7,1,3]]$ CSS code.


Quantum Hypergraph Product (QHP) codes:
A CSS code defined by the hypergraph product of two graphs, which
corresponds to two classical codes.

Quantum pin codes

Quantum XYZ product codes

Rotated surface codes

Shor's codes

Space-time codes: Multiple measurements will add a temporal dimension
to the code.

Steane codes: A $[[5,1,3]]$ code

Subsystem codes

Subsystem product codes

Subsystem hypergraph product codes

Surface codes: constructed from any tessellation of an arbitrary
surface or a higher-dimensional manifold. Generalization of the toric codes.

Surface-GKP codes: A concatenated code with GKP code as the inner
code, and surface code as the outer code.

Tensor network codes \cite{farrelly2020tensor}: Define stabilizer code
using a tensor, which maps the physical qubits to logical qubits.



Toric codes: The code is defined on a periodic square lattice, that
is, a torus. The check operators are weight-4 vertex operators and
plaquette operators. The logical operators are nontrivial cycles on
the torus.





\section{Contribution guide}
Contributions are welcome!
\begin{itemize}
\item add codes
\item add reference, original paper for construction is prefered
\item add one liner explanation, to let the general audience know what
  is it. We don't intend to teach the detail of the codes here. We
  suppose the audience are familiar with it, otherwise they can learn
  from the references.
\item classify the codes into categories
\item polish this latex document
\end{itemize}

\section{Similar projects}
Quantum Algorithm Zoo 
\url{https://quantumalgorithmzoo.org/}

Quantum protocol Zoo
\url{https://wiki.veriqloud.fr/index.php?title=Protocol_Library}

\bibliography{WeileiBibFile}
%\bibliography{lpp,qc_all,more_qc,ldpc,linalg,percol,sg}
\end{document}
